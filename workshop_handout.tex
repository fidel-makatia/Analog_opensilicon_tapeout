\documentclass[11pt]{article}

\usepackage[a4paper,margin=1in]{geometry}
\usepackage{amsmath}
\usepackage{amssymb}
\usepackage{enumitem}
\usepackage{hyperref}
\usepackage{listings}

\title{\textbf{3-Session Digital IC Tapeout Workshop}\\
Simplified 8-bit Microcontroller using SKY130}
\author{Instructor Notes + Student Handout}
\date{}

\begin{document}
\maketitle

\section{Workshop Objective}

This workshop introduces participants to a modern digital ASIC flow by taping out a simplified 8-bit microcontroller using open-source tools and an open CPU core.

Participants will:
\begin{itemize}
  \item Modify an existing CPU core (not design from scratch)
  \item Simulate and verify RTL
  \item Run synthesis, place-and-route, and DRC/LVS
  \item Export a GDS ready for fabrication
\end{itemize}

The emphasis is on \textbf{flow comprehension and tapeout success}, not performance optimization.

\section{Design Philosophy}

To fit within 3 sessions (2 hours each), the design is intentionally constrained:

\begin{itemize}
  \item Use an existing open CPU core
  \item Freeze the architecture
  \item Limit student changes to safe, local modifications
  \item Fully script backend flows
\end{itemize}

Students will \textbf{modify and tape out}, not architect a CPU.

\section{CPU Options}

Two supported cores are provided. The instructor selects one.

\subsection{Option A: PicoRV8 (Simplified RISC-V)}

\begin{itemize}
  \item Cut-down RISC-V-like core
  \item 8-bit datapath
  \item Hard-coded program ROM
  \item GPIO-only output
\end{itemize}

\subsection{Option B: 8-bit Accumulator Machine}

\begin{itemize}
  \item Custom educational CPU
  \item Accumulator-based ISA
  \item Minimal control logic
  \item Easiest to understand and debug
\end{itemize}

\noindent\textbf{Recommendation:} Option B for mixed audiences, Option A for RTL-savvy audiences.

\section{Frozen Architecture}

The following blocks are frozen and must not be modified:

\begin{itemize}
  \item Program counter
  \item Instruction decoder
  \item Clock/reset logic
  \item Top-level SoC wrapper
\end{itemize}

Students are allowed to modify:
\begin{itemize}
  \item ALU logic
  \item GPIO behavior
  \item Instruction behavior (within bounds)
  \item Program ROM contents
\end{itemize}

\section{Target Chip Characteristics}

\begin{itemize}
  \item Process: SKY130 (130 nm CMOS)
  \item Clock frequency: $\leq$ 5 MHz
  \item Area target: $\ll$ 1 mm$^2$
  \item Power: not constrained
\end{itemize}

\section{Toolchain Overview}

\begin{itemize}
  \item RTL simulation: Icarus Verilog
  \item Synthesis: Yosys
  \item Place \& Route: OpenROAD
  \item DRC/LVS: Magic + netgen
  \item Tapeout: SKY130 MPW or TinyTapeout
\end{itemize}

\section{Repository Structure}

Students are given a prepared Git repository with the following structure:

\begin{verbatim}
digital_tapeout_workshop/
├── rtl/
│   ├── core/
│   │   ├── alu.v
│   │   ├── regfile.v
│   │   └── control.v
│   ├── soc_top.v
│   └── gpio.v
├── tb/
│   └── soc_tb.v
├── rom/
│   └── program.hex
├── constraints/
│   └── sky130.sdc
├── flow/
│   ├── synth.tcl
│   ├── floorplan.tcl
│   └── pnr.tcl
├── outputs/
│   ├── netlist.v
│   └── chip.gds
└── README.md
\end{verbatim}

Only files in \texttt{rtl/core/} and \texttt{rom/} are modified by students.

\section{3-Session Lesson Plan (2 Hours Each)}

\subsection*{Session 1: RTL Understanding and Simulation}
\textbf{Tools: Verilog, Icarus Verilog}

\begin{itemize}
  \item CPU architecture walkthrough
  \item Instruction execution overview
  \item Modify ALU or GPIO logic
  \item Modify program ROM
  \item Run RTL simulation
\end{itemize}

\noindent\textbf{Deliverable:} Passing testbench

\subsection*{Session 2: Synthesis and Place \& Route}
\textbf{Tools: Yosys, OpenROAD}

\begin{itemize}
  \item RTL synthesis to gates
  \item Gate count inspection
  \item Floorplanning (scripted)
  \item Placement and routing
  \item Timing check (low-frequency)
\end{itemize}

\noindent\textbf{Deliverable:} Routed design

\subsection*{Session 3: Verification and Tapeout}
\textbf{Tools: OpenROAD, Magic, netgen}

\begin{itemize}
  \item DRC and LVS (scripted)
  \item GDS export
  \item Visual inspection of layout
  \item Tapeout submission
  \item Discussion of silicon risks
\end{itemize}

\noindent\textbf{Deliverable:} Final GDS

\section{Key Teaching Constraints}

\begin{itemize}
  \item No free-form RTL creation
  \item No clock gating
  \item No asynchronous logic
  \item No performance optimization
\end{itemize}

These constraints are enforced to guarantee tapeout success.

\section{Final Message to Participants}

\begin{quote}
``You are not building a product CPU.
You are learning how real chips become silicon.''
\end{quote}

\end{document}
